%% BioMed_Central_Tex_Template_v1.06
%%                                      %
%  bmc_article.tex            ver: 1.06 %
%                                       %


%%% additional documentclass options:
%  [doublespacing]
%  [linenumbers]   - put the line numbers on margins


%\documentclass[twocolumn]{bmcart}% uncomment this for twocolumn layout and comment line below
\documentclass{bmcart}

%%% Load packages
%\usepackage{amsthm,amsmath}
%\RequirePackage{natbib}
%\RequirePackage[authoryear]{natbib}% uncomment this for author-year bibliography
%\RequirePackage{hyperref}
\usepackage[utf8]{inputenc} %unicode support
%\usepackage[applemac]{inputenc} %applemac support if unicode package fails
%\usepackage[latin1]{inputenc} %UNIX support if unicode package fails


\def\includegraphic{}
\def\includegraphics{}


%%% Put your definitions there:
\startlocaldefs
\endlocaldefs
\textsl{}

%%% Begin ...
\begin{document}

%%% Start of article front matter
\begin{frontmatter}

\begin{fmbox}
\dochead{Investigación}


\title{Teletransportación}


\author[
   email={tomas.perrin.rivemar@gmail.com}
]{\inits{TPR}\fnm{Tomás} \snm{Perrín Rivemar}}



\begin{artnotes}
\end{artnotes}

\end{fmbox}% comment this for two column layout


\end{frontmatter}


\section*{Introducción}
Una forma de transporte de en sueños, esto es lo que parece la teletransportación ya que se habla de trasladarse de un punto a otro instantáneamente, algo que es imposible hoy en día.
Lo que esta investigación pretende desarrollar es la idea de una máquina capaz de hacer esto posible, así como dar un poco de información respecto a la teletransportación en la cultura popular (cinematografía, literatura, televisión, etc.), proyectos realizados en ésta área, y los logros que se han obtenido para el avance del tema.

\section*{Planteamiento del problema}
No es posible realizar la teletransportación actualmente ya que se carece de la tecnología y conocimientos necesarios, no se conocen todas las moléculas y el orden que estas llevan, y no se cuenta con la maquinaria necesaria para separar todas las pequeñas partes del cuerpo y volverlas a unir, entre otras limitantes. También, el objetivo de la teletransportación ha variado ligeramente dentro de algunas de las investigaciones más importantes.
\subsection*{}
La tecnología hoy en día avanza muy rápido así que se espera solucionar los problemas en un corto tiempo, si esto no sucede se tendría que seguir usando los métodos tradicionales para el transporte y continuar con la investigación teórica de la teletransportación.
Según leyes físicas conservativas, el teletransporte sería imposible, ya que, el teletransporte de un objeto de un lugar original a un nuevo lugar, debe mantener en todo momento su energía, si se transporta un objeto de un lugar con altura 0 y se desplaza a un lugar con altura distinta de 0 existiría una necesaria compensación de energía, la cual no podría ser calculada de manera certera; por motivos de esta índole se está tabulando la opción de la imposibilidad de teletransporte.

\section*{Objetivos}
Hacer una recapitulación de investigaciones y la teletransportación durante la historia de la humanidad.
Demostrar que la teletransportación no es imposible, pero se necesita más tecnología y una información completa del objeto que se desea transportar. 
También que futuras generaciones se puedan basar en esta investigación como base teórica para un proyectos a futuro y con esto abrir paso a medios de transporte más eficientes, así como rápidos y dar un gran avance para la humanidad.
Comparar algunas de las teorías conocidas para buscar la que se encuentre más cercana a obtener resultados que impacten a la ciencia y la sociedad.

\section*{Justificación}
Esta investigación demostrara que se pueden transportar cosas de una manera mucho más rápida y efectiva. Es importante seguir investigando e invirtiendo tiempo en este tema ya que será una tecnología revolucionaria en un futuro talvez no muy lejano que ayudara en una gran diversidad de áreas como en las de transporte.
\section*{Marco de referencia}
-	Científicos holandeses habrían descubierto la forma de teletransportar datos de un electrón a otro ubicado a tres metros de distancia. Con este experimento, demostrarían la existencia de la teletransportación cuántica, que importantes científicos como Einstein negaron. 
El experimento ha sido realizado por investigadores del Instituto Kavli de Nanociencia, en la Universidad Tecnológica de Delft, en Holanda. A pesar de que la teletransportación cuántica se había logrado anteriormente en el pasado, esta es la primera vez que se logran transferir de un punto a otro todos los datos. 
Si lograsen aumentar la distancia actual de tres metros, este descubrimiento cambiaría muchas cosas. De hecho, en el mundo de la informática permitiría que la transmisión de datos se realizase de forma instantánea, en tiempo real, lo cual cambiaría por completo la forma en la que interactuamos con los ordenadores. Para demostrar esta teoría, a lo largo de su experimento han trabajado con bits cuánticos, también conocidos como qubits. 
Estos qubits son unas partículas especiales que pueden tener varios valores simultáneamente. A lo largo de su estudio, los han separado a una distancia de tres metros y han observado que los datos de un electrón se transfieren de inmediato al otro. Los investigadores que han llevado a cabo el experimento, han subido un video a YouTube en el que ellos mismos se encargan de explicar en qué consiste exactamente este descubrimiento. 
En el video dejan claro que tan solo se pueden teletransportar datos, por lo que teletransportar personas de un lugar a otro sigue siendo, por el momento, cosa de ciencia ficción.
\subsection*{}
-	Parece que cada vez estamos más cerca de poder usar la tecnología del teletransporte, la cual solemos ver en películas y series como Stargate o Star Trek, entre muchas otras. Un grupo de científicos e investigadores de la Universidad de Tokio, que por cierto forman parte de un equipo de genios llamado grupo Furusawa, han logrado llevar a cabo el teletransporte, en forma totalmente exitosa, de algunos cubits (también llamados qubits, es decir bits cuánticos). Esto se ha logrado con la ayuda de una tecnología especial que es única en el mundo y que probablemente no entendamos cómo funciona. Por supuesto que este tipo de teletransportación todavía se encuentra en un nivel bastante “primitivo” por llamarlo de alguna forma. 
Todavía no podemos teletransportar personas ni cosas grandes ni nada por el estilo. Se estima que el teletransporte a nivel cuántico podría llegar a ser beneficioso para la computación cuántica, aunque no estamos del todo seguros de cómo podrían llegar a utilizarlo en dicho contexto. Por ahora, lo que nosotros estamos esperando es un sistema de teletransporte capaz de llevarnos de casa al trabajo en un segundo, o bien a cualquier otro lugar que necesitemos ir.
\subsection*{}
-	Recientemente, un grupo de investigadores de la Universidad de Ciencia y Tecnología de Hefei, en China, consiguieron la primera teletransportación documentada de un objeto macroscópico al llevar información cuántica de un conjunto de átomos a otro situado a 150 metros de distancia. 
Los físicos chinos, dirigidos por Xiao-Hui Bao, utilizaron átomos de rubidio en su experimento y, siguiendo el mismo principio, transportaron información cuántica entre dos átomos separados en 150 metros, apoyados en fotones enlazados. 
“Esto es interesante como la primera teletransportación de dos objetos de tamaño macroscópico en una distancia a escala macroscópica”, escribe el equipo de investigación en el artículo donde se publicaron los resultados. 

Xiao-Hui y su equipo buscan incrementar la probabilidad de éxito en cada evento de teletransportación, incrementando la cantidad de tiempo que el conjunto de átomos puede almacenar la información antes de perderla (actualmente este periodo dura solo 100 microsegundos) y crear una cadena de átomos que demuestre mejor el potencial de esta técnica de ruteo cuántico. 
El principal ámbito en que este desarrollo podría tener aplicación es en el llamado Internet cuántico, en el cual la información podría transmitirse de un punto a otro sin ser destruida en el proceso.
\section*{En la cultura popular}
•El cuento “El hombre sin un cuerpo” de Edward Page Mitchell, es quizás la narración más temprana de ciencia ficción sobre el concepto de teletransportación, fue escrita en 1877.
\subsection*{}
•En el cuento/película/opera “La Mosca” un hombre logra la teletransportación pero durante el proceso se fusiona con una mosca que entró a la cabina junto a el.
\subsection*{}
•En la radio-comedia “Guía del viajero intergaláctico” describen la teletransportación como “Tan divertida como una buena patada en la cabeza” ya que implica que tus átomos sean arrancados en un lugar y reacomodarlos en otra parte.
\subsection*{}
•En la serie de “Star Trek” o “Viaje a las estrellas” la icónica teletransportación se utilizó para evitar gastar en los efectos visuales de una nave aterrizando en un planeta distinto cada semana. El efecto del transportador se logró haciendo un simple fadeout del personaje junto con un poco de diamantina.
\section*{Hipótesis}
La creación de una máquina de teletransportación impactará en el mundo en gran medida en lo que respecta en transporte, rapidez y calidad en el servicio de varias cosas y hacer estas máquinas adecuadas para el uso diario.
\section*{Aspectos metodológicos}
Se utilizara la investigación explicativa (Cualitativa) para buscar las razones o causas que ocasionan este fenómeno y en qué condiciones se podría dar, usando el análisis y la síntesis para llegar a la mejor conclusión.
Desafortunadamente obtener información primaria de este tema (entrevistas, encuestas, sondeos, etc.) es algo muy complicado, por lo tanto solo se usaran fuentes secundarias de información, investigando en bibliotecas, revistas y páginas web que tengan información creíble y validada o de una investigación realizada.
\section*{Desarrollo del tema}
Investigaciones:
La investigación realizada confirmó la posibilidad de hacer una maquina capaz de transportar un objeto de un lado a otro instantáneamente, ya que si se logró mover qubits(bits cuánticos), es posible que en un futuro se puedan trasladar objetos aún mayores. Con el progreso de la tecnología, con la experimentación y la recolección de información se podrán abrir bastantes horizontes para este tema.
\subsection*{Sobre los qubits}
Un qubit o cubit (o sea bit cuántico) es un sistema cuántico con dos estados propios y que puede ser manipulado arbitrariamente. Se trata de un sistema que solo puede ser descrito correctamente mediante la mecánica cuántica, y que solamente tiene dos estados bien distinguibles mediante medidas físicas. También se entiende por qubit la información que contiene ese sistema cuántico de dos estados posibles. En esta acepción, el qubit es la unidad mínima y por lo tanto constitutiva de la teoría de la información cuántica. Es un concepto fundamental para la computación cuántica y para la criptografía cuántica, el análogo cuántico del bit en informática.
Su importancia radica en que la cantidad de información contenida en un qubit, y, en particular, la forma en que esta información puede ser manipulada, es fundamental y cualitativamente diferente de un bit clásico. Hay operaciones lógicas, por ejemplo, que son posibles en un qubit y no en un bit.
El concepto de qubit es abstracto y no lleva asociado un sistema físico concreto. En la práctica, se han preparado diferentes sistemas físicos que, en ciertas condiciones, pueden describirse como qubits o conjuntos de qubits. Los sistemas pueden ser de tamaño macroscópico, como un circuito superconductor, o microscópico, como un conjunto de iones suspendidos mediante campos eléctricos.
\subsection*{Sobre la computación cuántica}
La computación cuántica es un paradigma de computación distinto al de la computación clásica. Se basa en el uso de qubits en lugar de bits, y da lugar a nuevas puertas lógicas que hacen posibles nuevos algoritmos.
Una misma tarea puede tener diferente complejidad en computación clásica y en computación cuántica, lo que ha dado lugar a una gran expectación, ya que algunos problemas intratables pasan a ser tratables. Mientras que un computador clásico equivale a una máquina de Turing, un computador cuántico equivale a una máquina de Turing cuántica.
Uno de los obstáculos principales es el problema de la decoherencia cuántica, que causa la pérdida del carácter unitario (y, más específicamente, la reversibilidad) de los pasos del algoritmo cuántico. Los tiempos de decoherencia para los sistemas candidatos, en particular el tiempo de relajación transversal (en la terminología usada en la tecnología de resonancia magnética nuclear e imaginería por resonancia magnética) está típicamente entre nanosegundos y segundos, a temperaturas bajas. Las tasas de error son típicamente proporcionales a la razón entre tiempo de operación frente a tiempo de decoherencia, de forma que cualquier operación debe ser completada en un tiempo mucho más corto que el tiempo de decoherencia. Si la tasa de error es lo bastante baja, es posible usar eficazmente la corrección de errores cuántica, con lo cual sí serían posibles tiempos de cálculo más largos que el tiempo de decoherencia y, en principio, arbitrariamente largos.
\subsection*{El Cuerpo}
Si se analiza las células del cuerpo, se identifican cada una por separado y se crea una base de datos para cada una de ellas y del cerebro, luego de esto se descomponen y se aplica lo hecho con el qubit, luego se reacomoda y se recomponen las células, se estaría teletransportando un ser vivo. Esto se tendría que llevar a cabo instantáneamente ya que al tardar un poco más se corre el peligro de que el sujeto perdía la vida, ya que dejaría de funcionar el cerebro y por ende todas las partes del cuerpo.
\subsection*{La Máquina}
La máquina empleada deberá estar construida por un contenedor para el sujeto, tanto en la parte donde se envía y en el lugar al que desea llegar, el contenedor deberá contener un líquido capaz de separar las células del cuerpo (Como agua), una aparato que oscile instantáneamente a una velocidad extremadamente alta, una maquina capaz de dar lectura al cuerpo y convertirla en datos vivos, para así almacenarlos y enviarlos a su destino.
\subsection*{Conclusión}
Si se logra cumplir con cada una de las condiciones que se da a conocer, es posible que la teletransportación se lleva a cabo y no solo eso sino también existirá la posibilidad de hacer avances en la medicina, ya que si se investiga lo suficiente como para conocer perfectamente el cuerpo humano, se podría encontrar la solución a enfermedades que hoy en día no tienen cura como el cáncer, sida, etc. 
Los humanos tienden a soñar y los sueños son simples metas a alcanzar, hoy la teletransportación es un sueño y mañana será nuestra realidad.


\begin{backmatter}

 \begin{thebibliography}
 \bibitem{http://wwwhatsnew.com/2014/06/01/investigadores-descubren-como-teletransportar-datos-de-forma-instantanea/
 	 http://tecnomagazine.net/2013/09/14/investigadores-japoneses-creen-haber-logrado-la-teletransportacion-cuantica/
 	 http://pijamasurf.com/2012/11/fisicos-consiguen-la-primera-teletransportacion-de-un-objeto-macroscopico/
 	 http://arxiv.org/abs/1211.2892}
 \end{thebibliography}

\end{backmatter}
\end{document}
